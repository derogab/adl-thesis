\begin{abstract}
Il riconoscimento dell'attività umana (Human Activity Recognition, HAR) è un campo molto attivo della ricerca e molte 
tecniche di \textit{deep learning} sviluppate negli ultimi anni hanno dimostrato di essere affidabili per la 
classificazione delle attività di vita quotidiana (Activities of Daily Living, ADLs).

Malgrado le avanzate tecnologie messe in campo, spesso si incorre in limiti e problemi.
Alcuni di essi derivanti direttamente dalla non idealità del mondo reale in cui viviamo, altri invece 
da attribuire ad aspetti puramente organizzativi, di gestione e significato dei dati raccolti.

La seguente trattazione tenta di descrivere quanto svolto durante l'esperienza di stage presso l'Università degli Studi di Milano - Bicocca,
durante il quale si è cercato di sviluppare un classificatore di ADLs in grado di apprendere ed analizzare i dati inerziali ottenuti da un'applicazione Android.

\end{abstract}