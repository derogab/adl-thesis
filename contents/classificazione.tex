\chapter{Classificazione}
\label{chapter:classification}

Il cuore del progetto riguarda la classificazione delle attività mediante i dati ottenuti.

Ho optato per l'utilizzo di Keras di cui vediamo le fasi di \textit{training} e di \textit{prediction}.
\subsubsection{Keras}
Keras \cite{keras} è una libreria open-source per le reti neurali che astrae lo sviluppo rendendolo più comprensibile, 
pur mantenendo pieno supporto alle librerie di più basso livello (es. Tensorflow \cite{tensorflow}) su cui si basa.



\section{Apprendimento e Test}
Un classificatore basa le sue predizioni sui modelli che riesce a ricavare dall'insieme di informazioni che ha a disposizione.
Nel nostro caso l'insieme di queste informazioni è contenuto nei file con i valori generati e ricevuti dall'applicazione.


\subsection{Caricamento dei dati}
I file CSV presentano sono organizzati come nell'esempio visto in figura \ref{fig:example-dataset-csv-accelerometer}.

È necessario ricordare che i dati ottenuti da differenti sensori sono stati immagazzinati dal \textit{ricevitore} in diversi file CSV. 
Nel caso in esame sono quindi presenti due file, uno per l'accelerometro ed un secondo per il giroscopio.

Entrambi contengono la stessa tipologia di informazioni e sono strutturati in modo equivalente.
Per questo motivo nelle procedure seguenti considererò un singolo dataset, ricordando però di dover applicare tutti i 
passaggi indistintamente ad entrambi.

\subsubsection{Lettura del file}
L'intero dataset contiene un ampio numero di record, ognuno dei quali è composto dalle seguenti informazioni:
\begin{itemize}
    \item un identificativo che raggruppa i valori ottenuti da una singola esecuzione
    \item un indice crescente che ordina i record di un archivio
    \item i valori acquisiti dal sensore
    \item l'istante temporale di acquisizione
    \item la posizione del dispositivo
    \item la relativa attività
\end{itemize}
Come è possibile notate i valori contenuti in un singolo record corrispondono a quelli inviati dall'applicazione durante 
la raccolta dei dati sensoriali di apprendimento,
eccezion fatta per il tipo di sensore che è già stato usato per la divisione preliminare.

\begin{listing}[H] 
    \inputminted[frame=single,framesep=10pt]{python}{snippets/classifier/read_csv.py}
    \caption{Creazione del dataframe a partire dal file CSV}
\end{listing}


\subsection{Valutazione della posizione del dispositivo}
Comunemente la posizione del dispositivo di raccolta dati viene spesso sottovalutata durante lo sviluppo di tecniche 
per il riconoscimento delle attività. Tuttavia si tratta di un aspetto molto importante per la valutazione dei dati \cite{umafall}.

Dopo una prima valutazione che prevedeva l'utilizzo di questo dato come una semplice caratteristica informativa, ho deciso 
di dare ad esso un'importanza maggiore. 

Il maggior valore è derivato dalla decisione di partizionare i record presenti nel dataset in base alla relativa posizione. 
Tale scelta comporta la necessità di eseguire le procedure di \textit{train} e \textit{test} seguenti su tutte le diverse partizioni 
e conseguentemente la creazione di un modello indipendente per ognuna di esse.

\vspace{5mm} %5mm vertical space

Ai fini della trattazione proseguirò considerando solamente il gruppo di dati relativo ad una posizione 
tra quelle che ho personalmente impostato.

Ricordando che i dati sono già stati in precedenza partizionati in base al sensore di riferimento, è importante tener presente 
che tutte le seguenti procedure dovranno essere ripetute per ogni posizione e per ogni sensore.

\vspace{5mm} %5mm vertical space

Consideriamo quindi solo i dati \textit{accelerometrici} e \textit{"nella mano destra"}.


\subsection{Visualizzazione grafica dei dati}
\subsubsection{Suddivisione grafica}
Per ogni partizione di dati che si va a considerare è possibile ottenere una chiara visualizzazione grafica 
della suddivisione per attività dei dati presenti.
\begin{figure}[H]
    \centering
    \includegraphics[scale = 0.60]{assets/images/classifications/accelerometer/right_hand/activity-type-graph-right-hand-acc.png}
    \caption{Visualizzazione della suddivisione per attività}
\end{figure}

\subsubsection{Grafici delle attività}
Inoltre, di ogni attività è possibile utilizzare una visualizzazione grafica per meglio comprendere le differenze tra le attività anziché 
basarsi solamente sulla lettura dei valori.

\begin{figure}[H]
    \centering
    \includegraphics[scale = 0.45]{assets/images/classifications/accelerometer/right_hand/jumps-right-hand-acc.png}
    \caption{Dati accelerometrici durante l'attività "Salti" con il dispositivo nella mano destra}
\end{figure}
\begin{figure}[H]
    \centering
    \includegraphics[scale = 0.45]{assets/images/classifications/accelerometer/right_hand/run-right-hand-acc.png}
    \caption{Dati accelerometrici durante l'attività "Corsa" con il dispositivo nella mano destra}
\end{figure}
\begin{figure}[H]
    \centering
    \includegraphics[scale = 0.45]{assets/images/classifications/accelerometer/right_hand/walk-right-hand-acc.png}
    \caption{Dati accelerometrici durante l'attività "Camminata" con il dispositivo nella mano destra}
\end{figure}


\subsection{Apprendimento}
La mole di dati considerata necessita un partizionamento per differenziare i dati che saranno utilizzati per 
l'\textit{apprendimento} e quelli che saranno utilizzati per il \textit{test}.

È indispensabile verificare che le partizioni non abbiano sovrapposizioni se non si vuole ottenere una valutazione dell'efficienza falsata.

\vspace{5mm} %5mm vertical space

Personalmente ho scelto una semplice suddivisione che prevede l'utilizzo di $\frac{4}{5}$ dei dati per il \textit{train} e 
la parte restante ($\frac{1}{5}$) per il \textit{test}.


\subsection{Preparazione dei dati}
Ci si aspetta che, fornita una serie di caratteristiche (i tre assi x, y, z ed il valore temporale, 
la rete neurale dia in risposta un'etichetta rappresentate l'attività associata.

\vspace{5mm} %5mm vertical space

Dobbiamo quindi organizzare i dati in nostro possesso in modo da renderlo possibile.

\subsubsection{Trasformazione del valore temporale}
Una delle quattro caratteristiche che intendiamo utilizzare è il valore temporale.
I \textit{timestamp}s però rappresentano il tempo assoluto, 
ovvero il momento esatto di svolgimento dell'attività durante la raccolta dei dati.

Il momento esatto non è in alcun modo rilevante nella classificazione, ma a partire da questo è possibile ricavare il tempo trascorso 
tra l'acquisizione di una tripla di dati (x, y, z) e l'acquisizione di quella immediatamente successiva.
Posso presumere una somiglianza in tali distanze temporali durante lo svolgimento di una uguale attività.

\subsubsection{Normalizzazione dei dati}
La rete neurale accetta in ingresso valori compresi tra 0 e 1. Eseguo una semplice normalizzazione sui dati delle caratteristiche.
\begin{listing}[H] 
    \inputminted[frame=single,framesep=10pt]{python}{snippets/classifier/normalize_data.py}
    \caption{Banale normalizzazione dei dati}
\end{listing}

\subsubsection{Creazione dei segmenti e delle etichette}
La parte principale dell'intero adattamento risulta essere quella che suddivide i dati in un formato che possa 
realizzare l'associazione tra una serie di caratteristiche e una etichetta.

Per realizzare ciò i record sono presi a gruppi, anche sovrapposti (così come si vede in figura \ref{fig:create_segments_and_labels}). 
Ogni raggruppamento sarà caratterizzato da 
\begin{itemize}
    \item un segmento contenente i record con le sole caratteristiche
    \item l'etichetta più frequente
\end{itemize}

\begin{figure}[H]
  \centering
  \includesvg[scale = 0.8]{assets/graphs/create_segments_and_labels.svg}
  \caption{Creazione dei segmenti e delle etichette}
  \label{fig:create_segments_and_labels}
\end{figure}

\begin{figure}[H]
  \centering
  \includesvg[scale = 0.65]{assets/graphs/segments_and_labels.svg}
  \caption{Risultato dopo la creazione dei segmenti e delle etichette}
  \label{fig:segments_and_labels}
\end{figure}

Alla fine del processo ho quindi ottenuto una serie di segmenti di dati a cui posso singolarmente associare una determinata
etichetta identificativa.


\subsection{Creazione della rete neurale}
Una volta generati i dati nel formato supportato da \textit{keras} procedo alla creazione di 
una rete neurale che abbia
\begin{itemize}
    \item in input il formato dei dati appena generato
    \item 5 strati di 100 nodi connessi
    \item in output il calcolo di propabilità per ogni classe
\end{itemize}
\begin{listing}[H] 
    \inputminted[frame=single,framesep=10pt]{python}{snippets/classifier/dnn_create.py}
    \caption{Creazione della DNN}
\end{listing}
Per poi procedere all'apprendimento.
\begin{listing}[H] 
    \inputminted[frame=single,framesep=10pt]{python}{snippets/classifier/dnn_fit.py}
    \caption{Apprendimento della rete neurale}
\end{listing}


\subsection{Statistiche del modello}
Al termine dell'attività di apprendimento il modello è stato generato. Ed è possibile vedere un grafico 
riassuntivo dei risultati ottenuti.
\begin{figure}[H]
    \centering
    \includegraphics[scale = 0.60]{assets/images/classifications/accelerometer/right_hand/model-right-hand-acc.png}
    \caption{Statistiche del modello ottenuto}
\end{figure}


\subsection{Testing della rete neurale}
Il test della rete neurale creata avviene mediante l'uso della partizione di dati che avevamo precedentemente separato.

L'intento è quello di effettuare una predizione con i dati di test, di cui però si conoscono già i risultati. 
Sarà quindi immediato trovare l'efficienza del modello creato mediante un banale confronto tra i dati ipotizzati 
dalla rete neurale e i risultati corretti.

La qualità dei dati può essere visualizzata mediante una \textit{matrice di confusione} che fornisce una rappresentazione 
grafica del confronto appena descritto.

\subsubsection{Matrice di confusione}
La matrice di confusione è una tabella di rappresentazione dell'accuratezza di un modello di classificazione. 
In una matrice di confusione sono contrapposti i valori ipotizzati da un modello di classificazione e i valori reali 
di cui si conosceva il corretto risultato.

\begin{figure}[H]
    \centering
    \includegraphics[scale = 0.60]{assets/images/classifications/accelerometer/right_hand/confusion-matrix-right-hand-acc.png}
    \caption{Matrice di confusione del modello ottenuto}
\end{figure}


\section{Predizioni}
In maniera del tutto equivalente a quanto già visto durante l'apprendimento, per effettuare una predizione sfruttando
la rete neurale allenata forniremo in input una serie di segmenti e attenderemo che il classificatore ci fornisca in risposta
una serie di etichette ipotizzate.

